\begin{ludol}

\item \label{sec:ludol} Spillets navn er \textsc{LUDØL}.

\item Spillets hensikt er å gi studenter et meningsfyllt fritidstilbud.

\item Spillet skal kun ha én dommer. Hvis spillet etterhvert skulle utvikle seg
  mot kaotiske tilstander og anarki, eller hvis dommeren av andre årsaker skulle
  miste kontrollen, kan denne utpeke to linjedommere som hjelpemannskap.

\item \label{par:forkunnskaper} Dommeren plikter på forhånd å gjøre seg kjent
  med Ludøl-spillets paragrafer og bør selv ha gjennomgått Ludøl's prøvelser.

\item \label{par:dommer} Dommeren kan etter eget skjønn avvike fra disse
  paragrafer.

\item Spillet skal ha maksimalt fire ($4$) deltakere.

\item To eller flere deltakere kan ikke dele samme farge.

\item En deltaker kan godt spille flere farger.

\item Spillet følger alminnelige ludo-regler med de utvidelser som er nedfelt her.

\item Deltakerne kaster om å begynne. Spilleren med høyest terningkast starter
  og blir tildelt bokstav $A$, spilleren til venstre blir tildelt $B$, nestemann
  $C$ osv. For å få brikkene hjem, må de havne på riktig felt i
  avslutningsfeltet.

\item Som brikker benyttes glass som (mens de er i aktivt spill) skal være fylt
  med edelt Pilsner-Øl (\SI{4.5}{\percent}) eller sterkere drikke. Utenom
  spillebrikkene er det kun lov å ha èn åpnet enhet alkohol om gangen. All alkohol
  \emph{skal} inneholde normal mengde kullsyre.

\item Idet et glass blir slått hjem, skal spilleren som eier glasset tømme dette
  innen $10$ sekunder. De andre deltakerne plikter å telle høyt og tydelig fra
  $10$ til $0$.

\item \label{par:sperre} En spiller kan sperre andre fra å passere ved å ha to
  eller fire glass stående på samme felt. Merk at tre glass \emph{ikke} sperrer.
  Ved hjemslåing må alle glass tømmes. En spiller kan fritt passere sin egen
  sperre.

\item En spiller kan ikke slås hjem når han/hun står i sitt
  \textquote{startfelt}.

%Label skaper ankerpunkt til refferanser
\item \label{par:roros} Røros-konvensjonen gjelder under alle normale kast. For
  de uinvidde betyr dette at spilleren som kastrer terningen fritt kan velge om
  han/hun ønsker å benytte oversiden eller undersiden av terningen.

  % Presser inn en ny side, og lar den nye siden begynne 2.38cm høyere opp enn
  % normalt. Nødvendig for å få det på to sider.

 \ifnothandbok{\newpage %
  \vspace*{-1.38cm}} \thispagestyle{empty} % Fjerner sidetall

%Cref gir stor bokstav til refferanser. Henviser til røross-paragrafen.
\item \label{par:roros-2} \Cref{par:roros} trer ikke i kraft under følgende
  omstendigheter:

  \begin{ludol}

  \item Ved innledende kast om hvilken spiller som skal begynne.

  \item I det en brikke befinner seg i, eller flyttes inn i avslutningsfeltet.
    Merk at \cref{par:sekser-regelen} har prioritet over denne paragrafen.
    \ifhandbok{Overskrides av husregel 19}.

  \end{ludol}

\item Dommeren kan etter eget skjønn tildele en eller flere spillere
  straffepils. Det anbefales på det sterkeste at dette gjøres ved følgende
  tilfeller:

  \begin{ludol}

  \item En spiller kaster terningen slik at den faller på gulvet.

  \item Ved alle tilfeller der en spiller er ansvarlig for ødsling av edelt
    drikke.

  \item Hvis en spiller unnlater å benytte anledningen til å slå hjem en brikke
    av annen farge.

  \item Hvis en spiller ikke er tilstede idet det er dennes tur til å kaste
    terningen og nedtelling fra $10$ til $0$ er utført av de andre deltakerne.
    Dette medfører at man bør være rask under eventuelle toalettbesøk.

  \item Hvis en spiller er tilstede, men (etter nedtelling) likevel ikke er klar
    over at det hans/hennes tur til å kaste terningen.

  \item Ved ethvert forsøk på å kverulere over dommerens avgjørelse.

  \item Hvis en spiller viser liten evne til å vite hvilken farge som er ens
    egen.

  \item \label{subpar:passivt} Ved forsøk på passivt spill.

  \end{ludol}

\item \label{par:sekser-regelen} En spiller kan i henhold til \cref{par:roros}
  velge om han/hun vil benytte terningen som en éner ($1$) eller sekser ($6$)
  hvis et av disse tallene oppnås under kast. Velger spilleren én ($1$), flyttes
  glasset ett felt framover, og spilleren får ikke ekstra kast (han/hun kan
  imidlertid risikere å bli idømt straffepils i henhold til
  \cref{subpar:passivt}). Hvis spilleren velger å benytte terningen som en
  sekser ($6$), flyttes glasset seks felter fram (eller et nytt glass flyttes
  \textquote{ut}), og spilleren får ekstrakast. Fra og med tredje kast, må
  spilleren imidlertid velge seks ($6$) hvis terningen viser én ($1$) eller seks
  ($6$), og spilleren idømmes i tillegg én straffepils for hver sekser (éner)
  fra og med den tredje.

\item \label{par:offentliggjore} Den som for tredje gang ser seg nødt til å
  offentliggjøre sin middag, diskvalifiseres.

\item \label{par:avsluttes} Spillet avsluttes når:

  \begin{ludol}

  \item \label{subpar:hjem} En spiller har fått alle sine glass hjem.

  \item \label{subpar:tom} I det dommeren har skjenket all drikke.
    \ifhandbok{ALLAH AHKBAR}

  \item \label{subpar:middag} Alle untatt én er diskvalifisert i henhold til
    \cref{par:offentliggjore}.

  \end{ludol}

\item Hvis spillet er avsluttet i henhold til \cref{subpar:hjem}. eller
  \cref{subpar:middag}, er vinneren den person som fikk alle sine glass
  hjem/ikke ble diskvalifisert. Hvis spillet er avsluttet i henhold til
  \cref{subpar:tom}, telles alle glass sin avstand (i felter) fra startfeltet.
  For de glass som ennå er hjemme, trekkes det ti ($10$) felter fra. For de
  glass som allerede er kommet inn, legges ti ($10$) felter til summen. Vinneren
  er den spilleren som oppnår høyest sum.

\item Ved avslutning kan spillerne tømme sine respektive glass hvis de ønsker
  dette.

\item Hvis spillet er avsluttet i henhold til \cref{subpar:hjem} eller
  \ref{subpar:middag}, blir vinneren tilkjent resten av kassen, såfremt den
  tømmes i løpet av kvelden.

\end{ludol}